\section{Einleitung}
Testsätze mit ö, ü und ß.
\cite{Dum:Bei}

\begin{figure}[h]
    \centering
    \includegraphics[width=\textwidth]{image/transferleistung.jpg}
    \caption{Ein Beispielbild im Text.}
    \label{fig:tl-logo}
\end{figure}

\begin{table}[h]
    \centering
    \begin{tabular}{|p{0.25\textwidth}|p{0.25\textwidth}|p{0.25\textwidth}|}
        \hline
        Tabellenkopf 1 & Tabellenkopf 2 & Tabellenkopf 3 \\\hline\hline
        Daten 1 & Daten 2 & Daten 3 \\\hline
        \multicolumn{3}{|c|}{Zusammengefasste Spalten} \\\hline
    \end{tabular}
    \caption{Eine Beispieltabelle im Text.}
    \label{tab:example}
\end{table}

\begin{lstlisting}[caption=Ausgabe Hallo Welt, label=sysout, captionpos=b]
    System.out.println("Hallo Welt");
\end{lstlisting}


\begin{figure}[h!]
    \centering
    \begin{tikzpicture}[auto,node distance = 0.9cm]
        \node[entity](blatt){AntragsblattImpl}[grow = up, sibling distance = 4cm]
            child {node[key attribute]{id}};

        \node[relationship](rel1)[right = of blatt]{Hat};

        \node[entity](container)[right = of rel1]{TeilVorhabenContainer}[grow = up, sibling distance = 4cm]
            child {node[key attribute]{id}};

        \path(rel1) edge node {1} (blatt) edge node {1} (container);

        \node[relationship](rel2)[right = of container]{Hat};

        \node[entity](content)[right = of rel2]{TeilVorhabenImpl}[grow = up, sibling distance = 2cm]
            child {node[key attribute]{id}}
            child {node[attribute]{owner}};

        \path(rel2) edge node {1} (container) edge node {0-n} (content);
    \end{tikzpicture}
    \caption{Entity-Relationship-Diagramm für das Praxisbeispiel}\label{fig:case:ERD}
\end{figure}

\begin{figure}[h!]
    \centering
    
\begin{tikzpicture}
    \begin{class}[text width=4cm]{AntragsblattImpl}{0,0}
        \attribute{- id : Long}
        \attribute{- teilVorhabenContainer : TeilVorhabenContainer}
    \end{class}

    \begin{class}[text width=5cm]{TeilVorhabenContainer}{6.5,-0.2}
        \attribute{- id : Long}
        \attribute{- teilvorhaben : List<TeilVorhabenImpl>}
    \end{class}

    \begin{class}[text width=4cm]{TeilVorhabenImpl}{13,-0.65}
        \attribute{- id : Long}
    \end{class}

    \unidirectionalAssociation{AntragsblattImpl}{1}{}{TeilVorhabenContainer}{1}{}
    \unidirectionalAssociation{TeilVorhabenContainer}{1}{}{TeilVorhabenImpl}{0..*}{}
\end{tikzpicture}

    \caption{UML-Klassendiagramm für das Praxisbeispiel}\label{fig:case:UML}
\end{figure}
